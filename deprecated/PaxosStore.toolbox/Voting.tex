\batchmode %% Suppresses most terminal output.
\documentclass{article}
\usepackage{color}
\definecolor{boxshade}{gray}{0.85}
\setlength{\textwidth}{360pt}
\setlength{\textheight}{541pt}
\usepackage{latexsym}
\usepackage{ifthen}
% \usepackage{color}
%%%%%%%%%%%%%%%%%%%%%%%%%%%%%%%%%%%%%%%%%%%%%%%%%%%%%%%%%%%%%%%%%%%%%%%%%%%%%
% SWITCHES                                                                  %
%%%%%%%%%%%%%%%%%%%%%%%%%%%%%%%%%%%%%%%%%%%%%%%%%%%%%%%%%%%%%%%%%%%%%%%%%%%%%
\newboolean{shading} 
\setboolean{shading}{false}
\makeatletter
 %% this is needed only when inserted into the file, not when
 %% used as a package file.
%%%%%%%%%%%%%%%%%%%%%%%%%%%%%%%%%%%%%%%%%%%%%%%%%%%%%%%%%%%%%%%%%%%%%%%%%%%%%
%                                                                           %
% DEFINITIONS OF SYMBOL-PRODUCING COMMANDS                                  %
%                                                                           %
%    TLA+      LaTeX                                                        %
%    symbol    command                                                      %
%    ------    -------                                                      %
%    =>        \implies                                                     %
%    <:        \ltcolon                                                     %
%    :>        \colongt                                                     %
%    ==        \defeq                                                       %
%    ..        \dotdot                                                      %
%    ::        \coloncolon                                                  %
%    =|        \eqdash                                                      %
%    ++        \pp                                                          %
%    --        \mm                                                          %
%    **        \stst                                                        %
%    //        \slsl                                                        %
%    ^         \ct                                                          %
%    \A        \A                                                           %
%    \E        \E                                                           %
%    \AA       \AA                                                          %
%    \EE       \EE                                                          %
%%%%%%%%%%%%%%%%%%%%%%%%%%%%%%%%%%%%%%%%%%%%%%%%%%%%%%%%%%%%%%%%%%%%%%%%%%%%%
\newlength{\symlength}
\newcommand{\implies}{\Rightarrow}
\newcommand{\ltcolon}{\mathrel{<\!\!\mbox{:}}}
\newcommand{\colongt}{\mathrel{\!\mbox{:}\!\!>}}
\newcommand{\defeq}{\;\mathrel{\smash   %% keep this symbol from being too tall
    {{\stackrel{\scriptscriptstyle\Delta}{=}}}}\;}
\newcommand{\dotdot}{\mathrel{\ldotp\ldotp}}
\newcommand{\coloncolon}{\mathrel{::\;}}
\newcommand{\eqdash}{\mathrel = \joinrel \hspace{-.28em}|}
\newcommand{\pp}{\mathbin{++}}
\newcommand{\mm}{\mathbin{--}}
\newcommand{\stst}{*\!*}
\newcommand{\slsl}{/\!/}
\newcommand{\ct}{\hat{\hspace{.4em}}}
\newcommand{\A}{\forall}
\newcommand{\E}{\exists}
\renewcommand{\AA}{\makebox{$\raisebox{.05em}{\makebox[0pt][l]{%
   $\forall\hspace{-.517em}\forall\hspace{-.517em}\forall$}}%
   \forall\hspace{-.517em}\forall \hspace{-.517em}\forall\,$}}
\newcommand{\EE}{\makebox{$\raisebox{.05em}{\makebox[0pt][l]{%
   $\exists\hspace{-.517em}\exists\hspace{-.517em}\exists$}}%
   \exists\hspace{-.517em}\exists\hspace{-.517em}\exists\,$}}
\newcommand{\whileop}{\.{\stackrel
  {\mbox{\raisebox{-.3em}[0pt][0pt]{$\scriptscriptstyle+\;\,$}}}%
  {-\hspace{-.16em}\triangleright}}}

% Commands are defined to produce the upper-case keywords.
% Note that some have space after them.
\newcommand{\ASSUME}{\textsc{assume }}
\newcommand{\ASSUMPTION}{\textsc{assumption }}
\newcommand{\AXIOM}{\textsc{axiom }}
\newcommand{\BOOLEAN}{\textsc{boolean }}
\newcommand{\CASE}{\textsc{case }}
\newcommand{\CONSTANT}{\textsc{constant }}
\newcommand{\CONSTANTS}{\textsc{constants }}
\newcommand{\ELSE}{\settowidth{\symlength}{\THEN}%
   \makebox[\symlength][l]{\textsc{ else}}}
\newcommand{\EXCEPT}{\textsc{ except }}
\newcommand{\EXTENDS}{\textsc{extends }}
\newcommand{\FALSE}{\textsc{false}}
\newcommand{\IF}{\textsc{if }}
\newcommand{\IN}{\settowidth{\symlength}{\LET}%
   \makebox[\symlength][l]{\textsc{in}}}
\newcommand{\INSTANCE}{\textsc{instance }}
\newcommand{\LET}{\textsc{let }}
\newcommand{\LOCAL}{\textsc{local }}
\newcommand{\MODULE}{\textsc{module }}
\newcommand{\OTHER}{\textsc{other }}
\newcommand{\STRING}{\textsc{string}}
\newcommand{\THEN}{\textsc{ then }}
\newcommand{\THEOREM}{\textsc{theorem }}
\newcommand{\LEMMA}{\textsc{lemma }}
\newcommand{\PROPOSITION}{\textsc{proposition }}
\newcommand{\COROLLARY}{\textsc{corollary }}
\newcommand{\TRUE}{\textsc{true}}
\newcommand{\VARIABLE}{\textsc{variable }}
\newcommand{\VARIABLES}{\textsc{variables }}
\newcommand{\WITH}{\textsc{ with }}
\newcommand{\WF}{\textrm{WF}}
\newcommand{\SF}{\textrm{SF}}
\newcommand{\CHOOSE}{\textsc{choose }}
\newcommand{\ENABLED}{\textsc{enabled }}
\newcommand{\UNCHANGED}{\textsc{unchanged }}
\newcommand{\SUBSET}{\textsc{subset }}
\newcommand{\UNION}{\textsc{union }}
\newcommand{\DOMAIN}{\textsc{domain }}
% Added for tla2tex
\newcommand{\BY}{\textsc{by }}
\newcommand{\OBVIOUS}{\textsc{obvious }}
\newcommand{\HAVE}{\textsc{have }}
\newcommand{\QED}{\textsc{qed }}
\newcommand{\TAKE}{\textsc{take }}
\newcommand{\DEF}{\textsc{ def }}
\newcommand{\HIDE}{\textsc{hide }}
\newcommand{\RECURSIVE}{\textsc{recursive }}
\newcommand{\USE}{\textsc{use }}
\newcommand{\DEFINE}{\textsc{define }}
\newcommand{\PROOF}{\textsc{proof }}
\newcommand{\WITNESS}{\textsc{witness }}
\newcommand{\PICK}{\textsc{pick }}
\newcommand{\DEFS}{\textsc{defs }}
\newcommand{\PROVE}{\settowidth{\symlength}{\ASSUME}%
   \makebox[\symlength][l]{\textsc{prove}}\@s{-4.1}}%
  %% The \@s{-4.1) is a kludge added on 24 Oct 2009 [happy birthday, Ellen]
  %% so the correct alignment occurs if the user types
  %%   ASSUME X
  %%   PROVE  Y
  %% because it cancels the extra 4.1 pts added because of the 
  %% extra space after the PROVE.  This seems to works OK.
  %% However, the 4.1 equals Parameters.LaTeXLeftSpace(1) and
  %% should be changed if that method ever changes.
\newcommand{\SUFFICES}{\textsc{suffices }}
\newcommand{\NEW}{\textsc{new }}
\newcommand{\LAMBDA}{\textsc{lambda }}
\newcommand{\STATE}{\textsc{state }}
\newcommand{\ACTION}{\textsc{action }}
\newcommand{\TEMPORAL}{\textsc{temporal }}
\newcommand{\ONLY}{\textsc{only }}              %% added by LL on 2 Oct 2009
\newcommand{\OMITTED}{\textsc{omitted }}        %% added by LL on 31 Oct 2009
\newcommand{\@pfstepnum}[2]{\ensuremath{\langle#1\rangle}\textrm{#2}}
\newcommand{\bang}{\@s{1}\mbox{\small !}\@s{1}}
%% We should format || differently in PlusCal code than in TLA+ formulas.
\newcommand{\p@barbar}{\ifpcalsymbols
   \,\,\rule[-.25em]{.075em}{1em}\hspace*{.2em}\rule[-.25em]{.075em}{1em}\,\,%
   \else \,||\,\fi}
%% PlusCal keywords
\newcommand{\p@fair}{\textbf{fair }}
\newcommand{\p@semicolon}{\textbf{\,; }}
\newcommand{\p@algorithm}{\textbf{algorithm }}
\newcommand{\p@mmfair}{\textbf{-{}-fair }}
\newcommand{\p@mmalgorithm}{\textbf{-{}-algorithm }}
\newcommand{\p@assert}{\textbf{assert }}
\newcommand{\p@await}{\textbf{await }}
\newcommand{\p@begin}{\textbf{begin }}
\newcommand{\p@end}{\textbf{end }}
\newcommand{\p@call}{\textbf{call }}
\newcommand{\p@define}{\textbf{define }}
\newcommand{\p@do}{\textbf{ do }}
\newcommand{\p@either}{\textbf{either }}
\newcommand{\p@or}{\textbf{or }}
\newcommand{\p@goto}{\textbf{goto }}
\newcommand{\p@if}{\textbf{if }}
\newcommand{\p@then}{\,\,\textbf{then }}
\newcommand{\p@else}{\ifcsyntax \textbf{else } \else \,\,\textbf{else }\fi}
\newcommand{\p@elsif}{\,\,\textbf{elsif }}
\newcommand{\p@macro}{\textbf{macro }}
\newcommand{\p@print}{\textbf{print }}
\newcommand{\p@procedure}{\textbf{procedure }}
\newcommand{\p@process}{\textbf{process }}
\newcommand{\p@return}{\textbf{return}}
\newcommand{\p@skip}{\textbf{skip}}
\newcommand{\p@variable}{\textbf{variable }}
\newcommand{\p@variables}{\textbf{variables }}
\newcommand{\p@while}{\textbf{while }}
\newcommand{\p@when}{\textbf{when }}
\newcommand{\p@with}{\textbf{with }}
\newcommand{\p@lparen}{\textbf{(\,\,}}
\newcommand{\p@rparen}{\textbf{\,\,) }}   
\newcommand{\p@lbrace}{\textbf{\{\,\,}}   
\newcommand{\p@rbrace}{\textbf{\,\,\} }}

%%%%%%%%%%%%%%%%%%%%%%%%%%%%%%%%%%%%%%%%%%%%%%%%%%%%%%%%%
% REDEFINE STANDARD COMMANDS TO MAKE THEM FORMAT BETTER %
%                                                       %
% We redefine \in and \notin                            %
%%%%%%%%%%%%%%%%%%%%%%%%%%%%%%%%%%%%%%%%%%%%%%%%%%%%%%%%%
\renewcommand{\_}{\rule{.4em}{.06em}\hspace{.05em}}
\newlength{\equalswidth}
\let\oldin=\in
\let\oldnotin=\notin
\renewcommand{\in}{%
   {\settowidth{\equalswidth}{$\.{=}$}\makebox[\equalswidth][c]{$\oldin$}}}
\renewcommand{\notin}{%
   {\settowidth{\equalswidth}{$\.{=}$}\makebox[\equalswidth]{$\oldnotin$}}}


%%%%%%%%%%%%%%%%%%%%%%%%%%%%%%%%%%%%%%%%%%%%%%%%%%%%
%                                                  %
% HORIZONTAL BARS:                                 %
%                                                  %
%   \moduleLeftDash    |~~~~~~~~~~                 %
%   \moduleRightDash    ~~~~~~~~~~|                %
%   \midbar            |----------|                %
%   \bottombar         |__________|                %
%%%%%%%%%%%%%%%%%%%%%%%%%%%%%%%%%%%%%%%%%%%%%%%%%%%%
\newlength{\charwidth}\settowidth{\charwidth}{{\small\tt M}}
\newlength{\boxrulewd}\setlength{\boxrulewd}{.4pt}
\newlength{\boxlineht}\setlength{\boxlineht}{.5\baselineskip}
\newcommand{\boxsep}{\charwidth}
\newlength{\boxruleht}\setlength{\boxruleht}{.5ex}
\newlength{\boxruledp}\setlength{\boxruledp}{-\boxruleht}
\addtolength{\boxruledp}{\boxrulewd}
\newcommand{\boxrule}{\leaders\hrule height \boxruleht depth \boxruledp
                      \hfill\mbox{}}
\newcommand{\@computerule}{%
  \setlength{\boxruleht}{.5ex}%
  \setlength{\boxruledp}{-\boxruleht}%
  \addtolength{\boxruledp}{\boxrulewd}}

\newcommand{\bottombar}{\hspace{-\boxsep}%
  \raisebox{-\boxrulewd}[0pt][0pt]{\rule[.5ex]{\boxrulewd}{\boxlineht}}%
  \boxrule
  \raisebox{-\boxrulewd}[0pt][0pt]{%
      \rule[.5ex]{\boxrulewd}{\boxlineht}}\hspace{-\boxsep}\vspace{0pt}}

\newcommand{\moduleLeftDash}%
   {\hspace*{-\boxsep}%
     \raisebox{-\boxlineht}[0pt][0pt]{\rule[.5ex]{\boxrulewd
               }{\boxlineht}}%
    \boxrule\hspace*{.4em }}

\newcommand{\moduleRightDash}%
    {\hspace*{.4em}\boxrule
    \raisebox{-\boxlineht}[0pt][0pt]{\rule[.5ex]{\boxrulewd
               }{\boxlineht}}\hspace{-\boxsep}}%\vspace{.2em}

\newcommand{\midbar}{\hspace{-\boxsep}\raisebox{-.5\boxlineht}[0pt][0pt]{%
   \rule[.5ex]{\boxrulewd}{\boxlineht}}\boxrule\raisebox{-.5\boxlineht%
   }[0pt][0pt]{\rule[.5ex]{\boxrulewd}{\boxlineht}}\hspace{-\boxsep}}

%%%%%%%%%%%%%%%%%%%%%%%%%%%%%%%%%%%%%%%%%%%%%%%%%%%%%%%%%%%%%%%%%%%%%%%%%%%%%
% FORMATING COMMANDS                                                        %
%%%%%%%%%%%%%%%%%%%%%%%%%%%%%%%%%%%%%%%%%%%%%%%%%%%%%%%%%%%%%%%%%%%%%%%%%%%%%

%%%%%%%%%%%%%%%%%%%%%%%%%%%%%%%%%%%%%%%%%%%%%%%%%%%%%%%%%%%%%%%%%%%%%%%%%%%%%
% PLUSCAL SHADING                                                           %
%%%%%%%%%%%%%%%%%%%%%%%%%%%%%%%%%%%%%%%%%%%%%%%%%%%%%%%%%%%%%%%%%%%%%%%%%%%%%

% The TeX pcalshading switch is set on to cause PlusCal shading to be
% performed.  This changes the behavior of the following commands and
% environments to cause full-width shading to be performed on all lines.
% 
%   \tstrut \@x cpar mcom \@pvspace
% 
% The TeX pcalsymbols switch is turned on when typesetting a PlusCal algorithm,
% whether or not shading is being performed.  It causes symbols (other than
% parentheses and braces and PlusCal-only keywords) that should be typeset
% differently depending on whether they are in an algorithm to be typeset
% appropriately.  Currently, the only such symbol is "||".
%
% The TeX csyntax switch is turned on when typesetting a PlusCal algorithm in
% c-syntax.  This allows symbols to be format differently in the two syntaxes.
% The "else" keyword is the only one that is.

\newif\ifpcalshading \pcalshadingfalse
\newif\ifpcalsymbols \pcalsymbolsfalse
\newif\ifcsyntax     \csyntaxtrue

% The \@pvspace command makes a vertical space.  It uses \vspace
% except with \ifpcalshading, in which case it sets \pvcalvspace
% and the space is added by a following \@x command.
%
\newlength{\pcalvspace}\setlength{\pcalvspace}{0pt}%
\newcommand{\@pvspace}[1]{%
  \ifpcalshading
     \par\global\setlength{\pcalvspace}{#1}%
  \else
     \par\vspace{#1}%
  \fi
}

% The lcom environment was changed to set \lcomindent equal to
% the indentation it produces.  This length is used by the
% cpar environment to make shading extend for the full width
% of the line.  This assumes that lcom environments are not
% nested.  I hope TLATeX does not nest them.
%
\newlength{\lcomindent}%
\setlength{\lcomindent}{0pt}%

%\tstrut: A strut to produce inter-paragraph space in a comment.
%\rstrut: A strut to extend the bottom of a one-line comment so
%         there's no break in the shading between comments on 
%         successive lines.
\newcommand\tstrut%
  {\raisebox{\vshadelen}{\raisebox{-.25em}{\rule{0pt}{1.15em}}}%
   \global\setlength{\vshadelen}{0pt}}
\newcommand\rstrut{\raisebox{-.25em}{\rule{0pt}{1.15em}}%
 \global\setlength{\vshadelen}{0pt}}


% \.{op} formats operator op in math mode with empty boxes on either side.
% Used because TeX otherwise vary the amount of space it leaves around op.
\renewcommand{\.}[1]{\ensuremath{\mbox{}#1\mbox{}}}

% \@s{n} produces an n-point space
\newcommand{\@s}[1]{\hspace{#1pt}}           

% \@x{txt} starts a specification line in the beginning with txt
% in the final LaTeX source.
\newlength{\@xlen}
\newcommand\xtstrut%
  {\setlength{\@xlen}{1.05em}%
   \addtolength{\@xlen}{\pcalvspace}%
    \raisebox{\vshadelen}{\raisebox{-.25em}{\rule{0pt}{\@xlen}}}%
   \global\setlength{\vshadelen}{0pt}%
   \global\setlength{\pcalvspace}{0pt}}

\newcommand{\@x}[1]{\par
  \ifpcalshading
  \makebox[0pt][l]{\shadebox{\xtstrut\hspace*{\textwidth}}}%
  \fi
  \mbox{$\mbox{}#1\mbox{}$}}  

% \@xx{txt} continues a specification line with the text txt.
\newcommand{\@xx}[1]{\mbox{$\mbox{}#1\mbox{}$}}  

% \@y{cmt} produces a one-line comment.
\newcommand{\@y}[1]{\mbox{\footnotesize\hspace{.65em}%
  \ifthenelse{\boolean{shading}}{%
      \shadebox{#1\hspace{-\the\lastskip}\rstrut}}%
               {#1\hspace{-\the\lastskip}\rstrut}}}

% \@z{cmt} produces a zero-width one-line comment.
\newcommand{\@z}[1]{\makebox[0pt][l]{\footnotesize
  \ifthenelse{\boolean{shading}}{%
      \shadebox{#1\hspace{-\the\lastskip}\rstrut}}%
               {#1\hspace{-\the\lastskip}\rstrut}}}


% \@w{str} produces the TLA+ string "str".
\newcommand{\@w}[1]{\textsf{``{#1}''}}             


%%%%%%%%%%%%%%%%%%%%%%%%%%%%%%%%%%%%%%%%%%%%%%%%%%%%%%%%%%%%%%%%%%%%%%%%%%%%%
% SHADING                                                                   %
%%%%%%%%%%%%%%%%%%%%%%%%%%%%%%%%%%%%%%%%%%%%%%%%%%%%%%%%%%%%%%%%%%%%%%%%%%%%%
\def\graymargin{1}
  % The number of points of margin in the shaded box.

% \definecolor{boxshade}{gray}{.85}
% Defines the darkness of the shading: 1 = white, 0 = black
% Added by TLATeX only if needed.

% \shadebox{txt} puts txt in a shaded box.
\newlength{\templena}
\newlength{\templenb}
\newsavebox{\tempboxa}
\newcommand{\shadebox}[1]{{\setlength{\fboxsep}{\graymargin pt}%
     \savebox{\tempboxa}{#1}%
     \settoheight{\templena}{\usebox{\tempboxa}}%
     \settodepth{\templenb}{\usebox{\tempboxa}}%
     \hspace*{-\fboxsep}\raisebox{0pt}[\templena][\templenb]%
        {\colorbox{boxshade}{\usebox{\tempboxa}}}\hspace*{-\fboxsep}}}

% \vshade{n} makes an n-point inter-paragraph space, with
%  shading if the `shading' flag is true.
\newlength{\vshadelen}
\setlength{\vshadelen}{0pt}
\newcommand{\vshade}[1]{\ifthenelse{\boolean{shading}}%
   {\global\setlength{\vshadelen}{#1pt}}%
   {\vspace{#1pt}}}

\newlength{\boxwidth}
\newlength{\multicommentdepth}

%%%%%%%%%%%%%%%%%%%%%%%%%%%%%%%%%%%%%%%%%%%%%%%%%%%%%%%%%%%%%%%%%%%%%%%%%%%%%
% THE cpar ENVIRONMENT                                                      %
% ^^^^^^^^^^^^^^^^^^^^                                                      %
% The LaTeX input                                                           %
%                                                                           %
%   \begin{cpar}{pop}{nest}{isLabel}{d}{e}{arg6}                            %
%     XXXXXXXXXXXXXXX                                                       %
%     XXXXXXXXXXXXXXX                                                       %
%     XXXXXXXXXXXXXXX                                                       %
%   \end{cpar}                                                              %
%                                                                           %
% produces one of two possible results.  If isLabel is the letter "T",      %
% it produces the following, where [label] is the result of typesetting     %
% arg6 in an LR box, and d is is a number representing a distance in        %
% points.                                                                   %
%                                                                           %
%   prevailing |<-- d -->[label]<- e ->XXXXXXXXXXXXXXX                      %
%         left |                       XXXXXXXXXXXXXXX                      %
%       margin |                       XXXXXXXXXXXXXXX                      %
%                                                                           %
% If isLabel is the letter "F", then it produces                            %
%                                                                           %
%   prevailing |<-- d -->XXXXXXXXXXXXXXXXXXXXXXX                            %
%         left |         <- e ->XXXXXXXXXXXXXXXX                            %
%       margin |                XXXXXXXXXXXXXXXX                            %
%                                                                           %
% where d and e are numbers representing distances in points.               %
%                                                                           %
% The prevailing left margin is the one in effect before the most recent    %
% pop (argument 1) cpar environments with "T" as the nest argument, where   %
% pop is a number \geq 0.                                                   %
%                                                                           %
% If the nest argument is the letter "T", then the prevailing left          %
% margin is moved to the left of the second (and following) lines of        %
% X's.  Otherwise, the prevailing left margin is left unchanged.            %
%                                                                           %
% An \unnest{n} command moves the prevailing left margin to where it was    %
% before the most recent n cpar environments with "T" as the nesting        %
% argument.                                                                 %
%                                                                           %
% The environment leaves no vertical space above or below it, or between    %
% its paragraphs.  (TLATeX inserts the proper amount of vertical space.)    %
%%%%%%%%%%%%%%%%%%%%%%%%%%%%%%%%%%%%%%%%%%%%%%%%%%%%%%%%%%%%%%%%%%%%%%%%%%%%%

\newcounter{pardepth}
\setcounter{pardepth}{0}

% \setgmargin{txt} defines \gmarginN to be txt, where N is \roman{pardepth}.
% \thegmargin equals \gmarginN, where N is \roman{pardepth}.
\newcommand{\setgmargin}[1]{%
  \expandafter\xdef\csname gmargin\roman{pardepth}\endcsname{#1}}
\newcommand{\thegmargin}{\csname gmargin\roman{pardepth}\endcsname}
\newcommand{\gmargin}{0pt}

\newsavebox{\tempsbox}

\newlength{\@cparht}
\newlength{\@cpardp}
\newenvironment{cpar}[6]{%
  \addtocounter{pardepth}{-#1}%
  \ifthenelse{\boolean{shading}}{\par\begin{lrbox}{\tempsbox}%
                                 \begin{minipage}[t]{\linewidth}}{}%
  \begin{list}{}{%
     \edef\temp{\thegmargin}
     \ifthenelse{\equal{#3}{T}}%
       {\settowidth{\leftmargin}{\hspace{\temp}\footnotesize #6\hspace{#5pt}}%
        \addtolength{\leftmargin}{#4pt}}%
       {\setlength{\leftmargin}{#4pt}%
        \addtolength{\leftmargin}{#5pt}%
        \addtolength{\leftmargin}{\temp}%
        \setlength{\itemindent}{-#5pt}}%
      \ifthenelse{\equal{#2}{T}}{\addtocounter{pardepth}{1}%
                                 \setgmargin{\the\leftmargin}}{}%
      \setlength{\labelwidth}{0pt}%
      \setlength{\labelsep}{0pt}%
      \setlength{\itemindent}{-\leftmargin}%
      \setlength{\topsep}{0pt}%
      \setlength{\parsep}{0pt}%
      \setlength{\partopsep}{0pt}%
      \setlength{\parskip}{0pt}%
      \setlength{\itemsep}{0pt}
      \setlength{\itemindent}{#4pt}%
      \addtolength{\itemindent}{-\leftmargin}}%
   \ifthenelse{\equal{#3}{T}}%
      {\item[\tstrut\footnotesize \hspace{\temp}{#6}\hspace{#5pt}]
        }%
      {\item[\tstrut\hspace{\temp}]%
         }%
   \footnotesize}
 {\hspace{-\the\lastskip}\tstrut
 \end{list}%
  \ifthenelse{\boolean{shading}}%
          {\end{minipage}%
           \end{lrbox}%
           \ifpcalshading
             \setlength{\@cparht}{\ht\tempsbox}%
             \setlength{\@cpardp}{\dp\tempsbox}%
             \addtolength{\@cparht}{.15em}%
             \addtolength{\@cpardp}{.2em}%
             \addtolength{\@cparht}{\@cpardp}%
            % I don't know what's going on here.  I want to add a
            % \pcalvspace high shaded line, but I don't know how to
            % do it.  A little trial and error shows that the following
            % does a reasonable job approximating that, eliminating
            % the line if \pcalvspace is small.
            \addtolength{\@cparht}{\pcalvspace}%
             \ifdim \pcalvspace > .8em
               \addtolength{\pcalvspace}{-.2em}%
               \hspace*{-\lcomindent}%
               \shadebox{\rule{0pt}{\pcalvspace}\hspace*{\textwidth}}\par
               \global\setlength{\pcalvspace}{0pt}%
               \fi
             \hspace*{-\lcomindent}%
             \makebox[0pt][l]{\raisebox{-\@cpardp}[0pt][0pt]{%
                 \shadebox{\rule{0pt}{\@cparht}\hspace*{\textwidth}}}}%
             \hspace*{\lcomindent}\usebox{\tempsbox}%
             \par
           \else
             \shadebox{\usebox{\tempsbox}}\par
           \fi}%
           {}%
  }

%%%%%%%%%%%%%%%%%%%%%%%%%%%%%%%%%%%%%%%%%%%%%%%%%%%%%%%%%%%%%%%%%%%%%%%%%%%%%%
% THE ppar ENVIRONMENT                                                       %
% ^^^^^^^^^^^^^^^^^^^^                                                       %
% The environment                                                            %
%                                                                            %
%   \begin{ppar} ... \end{ppar}                                              %
%                                                                            %
% is equivalent to                                                           %
%                                                                            %
%   \begin{cpar}{0}{F}{F}{0}{0}{} ... \end{cpar}                             %
%                                                                            %
% The environment is put around each line of the output for a PlusCal        %
% algorithm.                                                                 %
%%%%%%%%%%%%%%%%%%%%%%%%%%%%%%%%%%%%%%%%%%%%%%%%%%%%%%%%%%%%%%%%%%%%%%%%%%%%%%
%\newenvironment{ppar}{%
%  \ifthenelse{\boolean{shading}}{\par\begin{lrbox}{\tempsbox}%
%                                 \begin{minipage}[t]{\linewidth}}{}%
%  \begin{list}{}{%
%     \edef\temp{\thegmargin}
%        \setlength{\leftmargin}{0pt}%
%        \addtolength{\leftmargin}{\temp}%
%        \setlength{\itemindent}{0pt}%
%      \setlength{\labelwidth}{0pt}%
%      \setlength{\labelsep}{0pt}%
%      \setlength{\itemindent}{-\leftmargin}%
%      \setlength{\topsep}{0pt}%
%      \setlength{\parsep}{0pt}%
%      \setlength{\partopsep}{0pt}%
%      \setlength{\parskip}{0pt}%
%      \setlength{\itemsep}{0pt}
%      \setlength{\itemindent}{0pt}%
%      \addtolength{\itemindent}{-\leftmargin}}%
%      \item[\tstrut\hspace{\temp}]}%
% {\hspace{-\the\lastskip}\tstrut
% \end{list}%
%  \ifthenelse{\boolean{shading}}{\end{minipage}  
%                                 \end{lrbox}%
%                                 \shadebox{\usebox{\tempsbox}}\par}{}%
%  }

 %%% TESTING
 \newcommand{\xtest}[1]{\par
 \makebox[0pt][l]{\shadebox{\xtstrut\hspace*{\textwidth}}}%
 \mbox{$\mbox{}#1\mbox{}$}} 

% \newcommand{\xxtest}[1]{\par
% \makebox[0pt][l]{\shadebox{\xtstrut{#1}\hspace*{\textwidth}}}%
% \mbox{$\mbox{}#1\mbox{}$}} 

%\newlength{\pcalvspace}
%\setlength{\pcalvspace}{0pt}
% \newlength{\xxtestlen}
% \setlength{\xxtestlen}{0pt}
% \newcommand\xtstrut%
%   {\setlength{\xxtestlen}{1.15em}%
%    \addtolength{\xxtestlen}{\pcalvspace}%
%     \raisebox{\vshadelen}{\raisebox{-.25em}{\rule{0pt}{\xxtestlen}}}%
%    \global\setlength{\vshadelen}{0pt}%
%    \global\setlength{\pcalvspace}{0pt}}
   
   %%%% TESTING
   
   %% The xcpar environment
   %%  Note: overloaded use of \pcalvspace for testing.
   %%
%   \newlength{\xcparht}%
%   \newlength{\xcpardp}%
   
%   \newenvironment{xcpar}[6]{%
%  \addtocounter{pardepth}{-#1}%
%  \ifthenelse{\boolean{shading}}{\par\begin{lrbox}{\tempsbox}%
%                                 \begin{minipage}[t]{\linewidth}}{}%
%  \begin{list}{}{%
%     \edef\temp{\thegmargin}%
%     \ifthenelse{\equal{#3}{T}}%
%       {\settowidth{\leftmargin}{\hspace{\temp}\footnotesize #6\hspace{#5pt}}%
%        \addtolength{\leftmargin}{#4pt}}%
%       {\setlength{\leftmargin}{#4pt}%
%        \addtolength{\leftmargin}{#5pt}%
%        \addtolength{\leftmargin}{\temp}%
%        \setlength{\itemindent}{-#5pt}}%
%      \ifthenelse{\equal{#2}{T}}{\addtocounter{pardepth}{1}%
%                                 \setgmargin{\the\leftmargin}}{}%
%      \setlength{\labelwidth}{0pt}%
%      \setlength{\labelsep}{0pt}%
%      \setlength{\itemindent}{-\leftmargin}%
%      \setlength{\topsep}{0pt}%
%      \setlength{\parsep}{0pt}%
%      \setlength{\partopsep}{0pt}%
%      \setlength{\parskip}{0pt}%
%      \setlength{\itemsep}{0pt}%
%      \setlength{\itemindent}{#4pt}%
%      \addtolength{\itemindent}{-\leftmargin}}%
%   \ifthenelse{\equal{#3}{T}}%
%      {\item[\xtstrut\footnotesize \hspace{\temp}{#6}\hspace{#5pt}]%
%        }%
%      {\item[\xtstrut\hspace{\temp}]%
%         }%
%   \footnotesize}
% {\hspace{-\the\lastskip}\tstrut
% \end{list}%
%  \ifthenelse{\boolean{shading}}{\end{minipage}  
%                                 \end{lrbox}%
%                                 \setlength{\xcparht}{\ht\tempsbox}%
%                                 \setlength{\xcpardp}{\dp\tempsbox}%
%                                 \addtolength{\xcparht}{.15em}%
%                                 \addtolength{\xcpardp}{.2em}%
%                                 \addtolength{\xcparht}{\xcpardp}%
%                                 \hspace*{-\lcomindent}%
%                                 \makebox[0pt][l]{\raisebox{-\xcpardp}[0pt][0pt]{%
%                                      \shadebox{\rule{0pt}{\xcparht}\hspace*{\textwidth}}}}%
%                                 \hspace*{\lcomindent}\usebox{\tempsbox}%
%                                 \par}{}%
%  }
%  
% \newlength{\xmcomlen}
%\newenvironment{xmcom}[1]{%
%  \setcounter{pardepth}{0}%
%  \hspace{.65em}%
%  \begin{lrbox}{\alignbox}\sloppypar%
%      \setboolean{shading}{false}%
%      \setlength{\boxwidth}{#1pt}%
%      \addtolength{\boxwidth}{-.65em}%
%      \begin{minipage}[t]{\boxwidth}\footnotesize
%      \parskip=0pt\relax}%
%       {\end{minipage}\end{lrbox}%
%       \setlength{\xmcomlen}{\textwidth}%
%       \addtolength{\xmcomlen}{-\wd\alignbox}%
%       \settodepth{\alignwidth}{\usebox{\alignbox}}%
%       \global\setlength{\multicommentdepth}{\alignwidth}%
%       \setlength{\boxwidth}{\alignwidth}%
%       \global\addtolength{\alignwidth}{-\maxdepth}%
%       \addtolength{\boxwidth}{.1em}%
%       \raisebox{0pt}[0pt][0pt]{%
%        \ifthenelse{\boolean{shading}}%
%          {\hspace*{-\xmcomlen}\shadebox{\rule[-\boxwidth]{0pt}{0pt}%
%                                 \hspace*{\xmcomlen}\usebox{\alignbox}}}%
%          {\usebox{\alignbox}}}%
%       \vspace*{\alignwidth}\pagebreak[0]\vspace{-\alignwidth}\par}
% % a multi-line comment, whose first argument is its width in points.
%  
   
%%%%%%%%%%%%%%%%%%%%%%%%%%%%%%%%%%%%%%%%%%%%%%%%%%%%%%%%%%%%%%%%%%%%%%%%%%%%%%
% THE lcom ENVIRONMENT                                                       %
% ^^^^^^^^^^^^^^^^^^^^                                                       %
% A multi-line comment with no text to its left is typeset in an lcom        % 
% environment, whose argument is a number representing the indentation       % 
% of the left margin, in points.  All the text of the comment should be      % 
% inside cpar environments.                                                  % 
%%%%%%%%%%%%%%%%%%%%%%%%%%%%%%%%%%%%%%%%%%%%%%%%%%%%%%%%%%%%%%%%%%%%%%%%%%%%%%
\newenvironment{lcom}[1]{%
  \setlength{\lcomindent}{#1pt} % Added for PlusCal handling.
  \par\vspace{.2em}%
  \sloppypar
  \setcounter{pardepth}{0}%
  \footnotesize
  \begin{list}{}{%
    \setlength{\leftmargin}{#1pt}
    \setlength{\labelwidth}{0pt}%
    \setlength{\labelsep}{0pt}%
    \setlength{\itemindent}{0pt}%
    \setlength{\topsep}{0pt}%
    \setlength{\parsep}{0pt}%
    \setlength{\partopsep}{0pt}%
    \setlength{\parskip}{0pt}}
    \item[]}%
  {\end{list}\vspace{.3em}\setlength{\lcomindent}{0pt}%
 }


%%%%%%%%%%%%%%%%%%%%%%%%%%%%%%%%%%%%%%%%%%%%%%%%%%%%%%%%%%%%%%%%%%%%%%%%%%%%%
% THE mcom ENVIRONMENT AND \mutivspace COMMAND                              %
% ^^^^^^^^^^^^^^^^^^^^^^^^^^^^^^^^^^^^^^^^^^^^                              %
%                                                                           %
% A part of the spec containing a right-comment of the form                 %
%                                                                           %
%      xxxx (*************)                                                 %
%      yyyy (* ccccccccc *)                                                 %
%      ...  (* ccccccccc *)                                                 %
%           (* ccccccccc *)                                                 %
%           (* ccccccccc *)                                                 %
%           (*************)                                                 %
%                                                                           %
% is typeset by                                                             %
%                                                                           %
%     XXXX \begin{mcom}{d}                                                  %
%            CCCC ... CCC                                                   %
%          \end{mcom}                                                       %
%     YYYY ...                                                              %
%     \multivspace{n}                                                       %
%                                                                           %
% where the number d is the width in points of the comment, n is the        %
% number of xxxx, yyyy, ...  lines to the left of the comment.              %
% All the text of the comment should be typeset in cpar environments.       %
%                                                                           %
% This puts the comment into a single box (so no page breaks can occur      %
% within it).  The entire box is shaded iff the shading flag is true.       %
%%%%%%%%%%%%%%%%%%%%%%%%%%%%%%%%%%%%%%%%%%%%%%%%%%%%%%%%%%%%%%%%%%%%%%%%%%%%%
\newlength{\xmcomlen}%
\newenvironment{mcom}[1]{%
  \setcounter{pardepth}{0}%
  \hspace{.65em}%
  \begin{lrbox}{\alignbox}\sloppypar%
      \setboolean{shading}{false}%
      \setlength{\boxwidth}{#1pt}%
      \addtolength{\boxwidth}{-.65em}%
      \begin{minipage}[t]{\boxwidth}\footnotesize
      \parskip=0pt\relax}%
       {\end{minipage}\end{lrbox}%
       \setlength{\xmcomlen}{\textwidth}%       % For PlusCal shading
       \addtolength{\xmcomlen}{-\wd\alignbox}%  % For PlusCal shading
       \settodepth{\alignwidth}{\usebox{\alignbox}}%
       \global\setlength{\multicommentdepth}{\alignwidth}%
       \setlength{\boxwidth}{\alignwidth}%      % For PlusCal shading
       \global\addtolength{\alignwidth}{-\maxdepth}%
       \addtolength{\boxwidth}{.1em}%           % For PlusCal shading
      \raisebox{0pt}[0pt][0pt]{%
        \ifthenelse{\boolean{shading}}%
          {\ifpcalshading
             \hspace*{-\xmcomlen}%
             \shadebox{\rule[-\boxwidth]{0pt}{0pt}\hspace*{\xmcomlen}%
                          \usebox{\alignbox}}%
           \else
             \shadebox{\usebox{\alignbox}}
           \fi
          }%
          {\usebox{\alignbox}}}%
       \vspace*{\alignwidth}\pagebreak[0]\vspace{-\alignwidth}\par}
 % a multi-line comment, whose first argument is its width in points.


% \multispace{n} produces the vertical space indicated by "|"s in 
% this situation
%   
%     xxxx (*************)
%     xxxx (* ccccccccc *)
%      |   (* ccccccccc *)
%      |   (* ccccccccc *)
%      |   (* ccccccccc *)
%      |   (*************)
%
% where n is the number of "xxxx" lines.
\newcommand{\multivspace}[1]{\addtolength{\multicommentdepth}{-#1\baselineskip}%
 \addtolength{\multicommentdepth}{1.2em}%
 \ifthenelse{\lengthtest{\multicommentdepth > 0pt}}%
    {\par\vspace{\multicommentdepth}\par}{}}

%\newenvironment{hpar}[2]{%
%  \begin{list}{}{\setlength{\leftmargin}{#1pt}%
%                 \addtolength{\leftmargin}{#2pt}%
%                 \setlength{\itemindent}{-#2pt}%
%                 \setlength{\topsep}{0pt}%
%                 \setlength{\parsep}{0pt}%
%                 \setlength{\partopsep}{0pt}%
%                 \setlength{\parskip}{0pt}%
%                 \addtolength{\labelsep}{0pt}}%
%  \item[]\footnotesize}{\end{list}}
%    %%%%%%%%%%%%%%%%%%%%%%%%%%%%%%%%%%%%%%%%%%%%%%%%%%%%%%%%%%%%%%%%%%%%%%%%
%    % Typesets a sequence of paragraphs like this:                         %
%    %                                                                      %
%    %      left |<-- d1 --> XXXXXXXXXXXXXXXXXXXXXXXX                       %
%    %    margin |           <- d2 -> XXXXXXXXXXXXXXX                       %
%    %           |                    XXXXXXXXXXXXXXX                       %
%    %           |                                                          %
%    %           |                    XXXXXXXXXXXXXXX                       %
%    %           |                    XXXXXXXXXXXXXXX                       %
%    %                                                                      %
%    % where d1 = #1pt and d2 = #2pt, but with no vspace between            %
%    % paragraphs.                                                          %
%    %%%%%%%%%%%%%%%%%%%%%%%%%%%%%%%%%%%%%%%%%%%%%%%%%%%%%%%%%%%%%%%%%%%%%%%%

%%%%%%%%%%%%%%%%%%%%%%%%%%%%%%%%%%%%%%%%%%%%%%%%%%%%%%%%%%%%%%%%%%%%%%
% Commands for repeated characters that produce dashes.              %
%%%%%%%%%%%%%%%%%%%%%%%%%%%%%%%%%%%%%%%%%%%%%%%%%%%%%%%%%%%%%%%%%%%%%%
% \raisedDash{wd}{ht}{thk} makes a horizontal line wd characters wide, 
% raised a distance ht ex's above the baseline, with a thickness of 
% thk em's.
\newcommand{\raisedDash}[3]{\raisebox{#2ex}{\setlength{\alignwidth}{.5em}%
  \rule{#1\alignwidth}{#3em}}}

% The following commands take a single argument n and produce the
% output for n repeated characters, as follows
%   \cdash:    -
%   \tdash:    ~
%   \ceqdash:  =
%   \usdash:   _
\newcommand{\cdash}[1]{\raisedDash{#1}{.5}{.04}}
\newcommand{\usdash}[1]{\raisedDash{#1}{0}{.04}}
\newcommand{\ceqdash}[1]{\raisedDash{#1}{.5}{.08}}
\newcommand{\tdash}[1]{\raisedDash{#1}{1}{.08}}

\newlength{\spacewidth}
\setlength{\spacewidth}{.2em}
\newcommand{\e}[1]{\hspace{#1\spacewidth}}
%% \e{i} produces space corresponding to i input spaces.


%% Alignment-file Commands

\newlength{\alignboxwidth}
\newlength{\alignwidth}
\newsavebox{\alignbox}

% \al{i}{j}{txt} is used in the alignment file to put "%{i}{j}{wd}"
% in the log file, where wd is the width of the line up to that point,
% and txt is the following text.
\newcommand{\al}[3]{%
  \typeout{\%{#1}{#2}{\the\alignwidth}}%
  \cl{#3}}

%% \cl{txt} continues a specification line in the alignment file
%% with text txt.
\newcommand{\cl}[1]{%
  \savebox{\alignbox}{\mbox{$\mbox{}#1\mbox{}$}}%
  \settowidth{\alignboxwidth}{\usebox{\alignbox}}%
  \addtolength{\alignwidth}{\alignboxwidth}%
  \usebox{\alignbox}}

% \fl{txt} in the alignment file begins a specification line that
% starts with the text txt.
\newcommand{\fl}[1]{%
  \par
  \savebox{\alignbox}{\mbox{$\mbox{}#1\mbox{}$}}%
  \settowidth{\alignwidth}{\usebox{\alignbox}}%
  \usebox{\alignbox}}



  
%%%%%%%%%%%%%%%%%%%%%%%%%%%%%%%%%%%%%%%%%%%%%%%%%%%%%%%%%%%%%%%%%%%%%%%%%%%%%
% Ordinarily, TeX typesets letters in math mode in a special math italic    %
% font.  This makes it typeset "it" to look like the product of the         %
% variables i and t, rather than like the word "it".  The following         %
% commands tell TeX to use an ordinary italic font instead.                 %
%%%%%%%%%%%%%%%%%%%%%%%%%%%%%%%%%%%%%%%%%%%%%%%%%%%%%%%%%%%%%%%%%%%%%%%%%%%%%
\ifx\documentclass\undefined
\else
  \DeclareSymbolFont{tlaitalics}{\encodingdefault}{cmr}{m}{it}
  \let\itfam\symtlaitalics
\fi

\makeatletter
\newcommand{\tlx@c}{\c@tlx@ctr\advance\c@tlx@ctr\@ne}
\newcounter{tlx@ctr}
\c@tlx@ctr=\itfam \multiply\c@tlx@ctr"100\relax \advance\c@tlx@ctr "7061\relax
\mathcode`a=\tlx@c \mathcode`b=\tlx@c \mathcode`c=\tlx@c \mathcode`d=\tlx@c
\mathcode`e=\tlx@c \mathcode`f=\tlx@c \mathcode`g=\tlx@c \mathcode`h=\tlx@c
\mathcode`i=\tlx@c \mathcode`j=\tlx@c \mathcode`k=\tlx@c \mathcode`l=\tlx@c
\mathcode`m=\tlx@c \mathcode`n=\tlx@c \mathcode`o=\tlx@c \mathcode`p=\tlx@c
\mathcode`q=\tlx@c \mathcode`r=\tlx@c \mathcode`s=\tlx@c \mathcode`t=\tlx@c
\mathcode`u=\tlx@c \mathcode`v=\tlx@c \mathcode`w=\tlx@c \mathcode`x=\tlx@c
\mathcode`y=\tlx@c \mathcode`z=\tlx@c
\c@tlx@ctr=\itfam \multiply\c@tlx@ctr"100\relax \advance\c@tlx@ctr "7041\relax
\mathcode`A=\tlx@c \mathcode`B=\tlx@c \mathcode`C=\tlx@c \mathcode`D=\tlx@c
\mathcode`E=\tlx@c \mathcode`F=\tlx@c \mathcode`G=\tlx@c \mathcode`H=\tlx@c
\mathcode`I=\tlx@c \mathcode`J=\tlx@c \mathcode`K=\tlx@c \mathcode`L=\tlx@c
\mathcode`M=\tlx@c \mathcode`N=\tlx@c \mathcode`O=\tlx@c \mathcode`P=\tlx@c
\mathcode`Q=\tlx@c \mathcode`R=\tlx@c \mathcode`S=\tlx@c \mathcode`T=\tlx@c
\mathcode`U=\tlx@c \mathcode`V=\tlx@c \mathcode`W=\tlx@c \mathcode`X=\tlx@c
\mathcode`Y=\tlx@c \mathcode`Z=\tlx@c
\makeatother

%%%%%%%%%%%%%%%%%%%%%%%%%%%%%%%%%%%%%%%%%%%%%%%%%%%%%%%%%%
%                THE describe ENVIRONMENT                %
%%%%%%%%%%%%%%%%%%%%%%%%%%%%%%%%%%%%%%%%%%%%%%%%%%%%%%%%%%
%
%
% It is like the description environment except it takes an argument
% ARG that should be the text of the widest label.  It adjusts the
% indentation so each item with label LABEL produces
%%      LABEL             blah blah blah
%%      <- width of ARG ->blah blah blah
%%                        blah blah blah
\newenvironment{describe}[1]%
   {\begin{list}{}{\settowidth{\labelwidth}{#1}%
            \setlength{\labelsep}{.5em}%
            \setlength{\leftmargin}{\labelwidth}% 
            \addtolength{\leftmargin}{\labelsep}%
            \addtolength{\leftmargin}{\parindent}%
            \def\makelabel##1{\rm ##1\hfill}}%
            \setlength{\topsep}{0pt}}%% 
                % Sets \topsep to 0 to reduce vertical space above
                % and below embedded displayed equations
   {\end{list}}

%   For tlatex.TeX
\usepackage{verbatim}
\makeatletter
\def\tla{\let\%\relax%
         \@bsphack
         \typeout{\%{\the\linewidth}}%
             \let\do\@makeother\dospecials\catcode`\^^M\active
             \let\verbatim@startline\relax
             \let\verbatim@addtoline\@gobble
             \let\verbatim@processline\relax
             \let\verbatim@finish\relax
             \verbatim@}
\let\endtla=\@esphack

\let\pcal=\tla
\let\endpcal=\endtla
\let\ppcal=\tla
\let\endppcal=\endtla

% The tlatex environment is used by TLATeX.TeX to typeset TLA+.
% TLATeX.TLA starts its files by writing a \tlatex command.  This
% command/environment sets \parindent to 0 and defines \% to its
% standard definition because the writing of the log files is messed up
% if \% is defined to be something else.  It also executes
% \@computerule to determine the dimensions for the TLA horizonatl
% bars.
\newenvironment{tlatex}{\@computerule%
                        \setlength{\parindent}{0pt}%
                       \makeatletter\chardef\%=`\%}{}


% The notla environment produces no output.  You can turn a 
% tla environment to a notla environment to prevent tlatex.TeX from
% re-formatting the environment.

\def\notla{\let\%\relax%
         \@bsphack
             \let\do\@makeother\dospecials\catcode`\^^M\active
             \let\verbatim@startline\relax
             \let\verbatim@addtoline\@gobble
             \let\verbatim@processline\relax
             \let\verbatim@finish\relax
             \verbatim@}
\let\endnotla=\@esphack

\let\nopcal=\notla
\let\endnopcal=\endnotla
\let\noppcal=\notla
\let\endnoppcal=\endnotla

%%%%%%%%%%%%%%%%%%%%%%%% end of tlatex.sty file %%%%%%%%%%%%%%%%%%%%%%% 
% last modified on Fri  3 August 2012 at 14:23:49 PST by lamport

\begin{document}
\tlatex
\setboolean{shading}{true}
 \@x{\makebox[0pt][r]{\scriptsize 1\hspace{1em}}}\moduleLeftDash\@xx{
 {\MODULE} Voting}\moduleRightDash\@xx{}%
\@x{\makebox[0pt][r]{\scriptsize 2\hspace{1em}} {\EXTENDS} Sets}%
\@x{\makebox[0pt][r]{\scriptsize 3\hspace{1em}}}\midbar\@xx{}%
 \@x{\makebox[0pt][r]{\scriptsize 4\hspace{1em}} {\CONSTANT} Value ,\,
 Acceptor ,\, Quorum}%
\@pvspace{8.0pt}%
 \@x{\makebox[0pt][r]{\scriptsize 6\hspace{1em}} {\ASSUME} QuorumAssumption
 \.{\defeq}}%
 \@x{\makebox[0pt][r]{\scriptsize 7\hspace{1em}}\@s{16.4} \.{\land}\@s{10.73}
 \A\, Q \.{\in} Quorum \.{:} Q \.{\subseteq} Acceptor}%
 \@x{\makebox[0pt][r]{\scriptsize 8\hspace{1em}}\@s{16.4} \.{\land}\@s{10.73}
 \A\, Q1 ,\, Q2 \.{\in} Quorum \.{:} Q1 \.{\cap} Q2 \.{\neq} \{ \}}%
\@pvspace{8.0pt}%
 \@x{\makebox[0pt][r]{\scriptsize 10\hspace{1em}} {\THEOREM} QuorumNonEmpty
 \.{\defeq} \A\, Q \.{\in} Quorum \.{:} Q \.{\neq} \{ \}}%
\@x{\makebox[0pt][r]{\scriptsize 11\hspace{1em}} {\BY} QuorumAssumption}%
\@pvspace{8.0pt}%
\@x{\makebox[0pt][r]{\scriptsize 13\hspace{1em}} Ballot \.{\defeq} Nat}%
\@x{\makebox[0pt][r]{\scriptsize 14\hspace{1em}}}\midbar\@xx{}%
 \@x{\makebox[0pt][r]{\scriptsize 15\hspace{1em}} {\VARIABLES} votes ,\,
 maxBal}%
\@pvspace{8.0pt}%
 \@x{\makebox[0pt][r]{\scriptsize 17\hspace{1em}} TypeOK \.{\defeq} \.{\land}
 votes \.{\in} [ Acceptor \.{\rightarrow} {\SUBSET} ( Ballot \.{\times} Value
 ) ]}%
 \@x{\makebox[0pt][r]{\scriptsize 18\hspace{1em}}\@s{56.14} \.{\land} maxBal
 \.{\in} [ Acceptor \.{\rightarrow} Ballot \.{\cup} \{ \.{-} 1 \} ]}%
\@x{\makebox[0pt][r]{\scriptsize 19\hspace{1em}}}\midbar\@xx{}%
 \@x{\makebox[0pt][r]{\scriptsize 20\hspace{1em}} VotedFor ( a ,\, b ,\, v )
 \.{\defeq} {\langle} b ,\, v {\rangle} \.{\in} votes [ a ]}%
\@pvspace{8.0pt}%
 \@x{\makebox[0pt][r]{\scriptsize 22\hspace{1em}} DidNotVoteAt ( a ,\, b )
 \.{\defeq} \A\, v \.{\in} Value \.{:} {\lnot} VotedFor ( a ,\, b ,\, v )}%
\@pvspace{8.0pt}%
 \@x{\makebox[0pt][r]{\scriptsize 24\hspace{1em}} ShowsSafeAt ( Q ,\, b ,\, v
 ) \.{\defeq}}%
 \@x{\makebox[0pt][r]{\scriptsize 25\hspace{1em}}\@s{8.2} \.{\land} \A\, a
 \.{\in} Q \.{:} maxBal [ a ] \.{\geq} b}%
\@y{\@s{0}%
 have promised
}%
\@xx{}%
 \@x{\makebox[0pt][r]{\scriptsize 26\hspace{1em}}\@s{8.2} \.{\land} \E\,
 c\@s{0.71} \.{\in}\@s{11.16} \.{-} 1 \.{\dotdot} ( b \.{-} 1 ) \.{:}}%
 \@x{\makebox[0pt][r]{\scriptsize 27\hspace{1em}}\@s{23.41} \.{\land} ( c
 \.{\neq} \.{-} 1 ) \.{\implies} \E\, a \.{\in} Q \.{:} VotedFor ( a ,\, c
 ,\, v )}%
 \@x{\makebox[0pt][r]{\scriptsize 28\hspace{1em}}\@s{23.41} \.{\land} \A\, d
 \.{\in} ( c \.{+} 1 ) \.{\dotdot} ( b \.{-} 1 ) ,\, a \.{\in} Q \.{:}
 DidNotVoteAt ( a ,\, d )}%
\@x{\makebox[0pt][r]{\scriptsize 29\hspace{1em}}}\midbar\@xx{}%
\@x{\makebox[0pt][r]{\scriptsize 30\hspace{1em}} Init \.{\defeq}}%
 \@x{\makebox[0pt][r]{\scriptsize 31\hspace{1em}}\@s{16.4} \.{\land} votes
 \.{=} [ a \.{\in} Acceptor \.{\mapsto} \{ \} ]}%
 \@x{\makebox[0pt][r]{\scriptsize 32\hspace{1em}}\@s{16.4} \.{\land} maxBal
 \.{=} [ a \.{\in} Acceptor \.{\mapsto} \.{-} 1 ]}%
\@pvspace{8.0pt}%
 \@x{\makebox[0pt][r]{\scriptsize 34\hspace{1em}} IncreaseMaxBal ( a ,\, b )
 \.{\defeq}}%
 \@x{\makebox[0pt][r]{\scriptsize 35\hspace{1em}}\@s{8.2} \.{\land} b \.{>}
 maxBal [ a ]}%
 \@x{\makebox[0pt][r]{\scriptsize 36\hspace{1em}}\@s{8.2} \.{\land} maxBal
 \.{'} \.{=} [ maxBal {\EXCEPT} {\bang} [ a ] \.{=} b ]}%
\@y{\@s{0}%
 make promise
}%
\@xx{}%
 \@x{\makebox[0pt][r]{\scriptsize 37\hspace{1em}}\@s{8.2} \.{\land}
 {\UNCHANGED} votes}%
\@pvspace{8.0pt}%
 \@x{\makebox[0pt][r]{\scriptsize 39\hspace{1em}} VoteFor ( a ,\, b ,\, v )
 \.{\defeq}}%
 \@x{\makebox[0pt][r]{\scriptsize 40\hspace{1em}}\@s{16.4} \.{\land}\@s{8.35}
 maxBal [ a ] \.{\leq} b}%
\@y{\@s{0}%
 keep promise
}%
\@xx{}%
 \@x{\makebox[0pt][r]{\scriptsize 41\hspace{1em}}\@s{16.4} \.{\land}\@s{8.35}
 \A\, vt \.{\in} votes [ a ] \.{:} vt [ 1 ] \.{\neq} b}%
 \@x{\makebox[0pt][r]{\scriptsize 42\hspace{1em}}\@s{16.4} \.{\land}\@s{8.35}
 \A\, c \.{\in} Acceptor \.{\,\backslash\,} \{ a \} \.{:}}%
 \@x{\makebox[0pt][r]{\scriptsize 43\hspace{1em}}\@s{44.06} \A\, vt \.{\in}
 votes [ c ] \.{:} ( vt [ 1 ] \.{=} b ) \.{\implies} ( vt [ 2 ] \.{=} v )}%
 \@x{\makebox[0pt][r]{\scriptsize 44\hspace{1em}}\@s{16.4} \.{\land}\@s{8.35}
 \E\, Q \.{\in} Quorum \.{:} ShowsSafeAt ( Q ,\, b ,\, v )}%
\@y{\@s{0}%
 safe to vote
}%
\@xx{}%
 \@x{\makebox[0pt][r]{\scriptsize 45\hspace{1em}}\@s{16.4} \.{\land}\@s{8.35}
 votes \.{'} \.{=} [ votes {\EXCEPT} {\bang} [ a ] \.{=} votes [ a ] \.{\cup}
 \{ {\langle} b ,\, v {\rangle} \} ]}%
\@y{\@s{0}%
 vote
}%
\@xx{}%
 \@x{\makebox[0pt][r]{\scriptsize 46\hspace{1em}}\@s{16.4} \.{\land}\@s{8.35}
 maxBal \.{'} \.{=} [ maxBal {\EXCEPT} {\bang} [ a ] \.{=} b ]}%
\@y{\@s{0}%
 make promise
}%
\@xx{}%
\@x{\makebox[0pt][r]{\scriptsize 47\hspace{1em}}}\midbar\@xx{}%
\@x{\makebox[0pt][r]{\scriptsize 48\hspace{1em}} Next \.{\defeq}}%
 \@x{\makebox[0pt][r]{\scriptsize 49\hspace{1em}}\@s{16.4} \E\, a \.{\in}
 Acceptor ,\, b \.{\in} Ballot \.{:}}%
 \@x{\makebox[0pt][r]{\scriptsize 50\hspace{1em}}\@s{27.72} \.{\lor}
 IncreaseMaxBal ( a ,\, b )}%
 \@x{\makebox[0pt][r]{\scriptsize 51\hspace{1em}}\@s{27.72} \.{\lor} \E\, v
 \.{\in} Value \.{:} VoteFor ( a ,\, b ,\, v )}%
\@pvspace{8.0pt}%
 \@x{\makebox[0pt][r]{\scriptsize 53\hspace{1em}} Spec \.{\defeq} Init
 \.{\land} {\Box} [ Next ]_{ {\langle} votes ,\, maxBal {\rangle}}}%
\@x{\makebox[0pt][r]{\scriptsize 54\hspace{1em}}}\midbar\@xx{}%
 \@x{\makebox[0pt][r]{\scriptsize 55\hspace{1em}} ChosenAt ( b ,\, v )
 \.{\defeq}}%
 \@x{\makebox[0pt][r]{\scriptsize 56\hspace{1em}}\@s{16.4} \E\, Q \.{\in}
 Quorum \.{:} \A\, a \.{\in} Q \.{:} VotedFor ( a ,\, b ,\, v )}%
\@pvspace{8.0pt}%
 \@x{\makebox[0pt][r]{\scriptsize 58\hspace{1em}} chosen \.{\defeq} \{ v
 \.{\in} Value \.{:} \E\, b \.{\in} Ballot \.{:} ChosenAt ( b ,\, v ) \}}%
\@pvspace{8.0pt}%
 \@x{\makebox[0pt][r]{\scriptsize 60\hspace{1em}} Consistency \.{\defeq}
 chosen \.{=} \{ \} \.{\lor} \E\, v \.{\in} Value \.{:} chosen \.{=} \{ v \}}%
\@y{\@s{0}%
 \ensuremath{Cardinality(chosen) \.{\leq} 1
}}%
\@xx{}%
\@x{\makebox[0pt][r]{\scriptsize 61\hspace{1em}}}\midbar\@xx{}%
 \@x{\makebox[0pt][r]{\scriptsize 62\hspace{1em}} CannotVoteAt ( a ,\, b )
 \.{\defeq}}%
 \@x{\makebox[0pt][r]{\scriptsize 63\hspace{1em}}\@s{16.4} \.{\land} maxBal [
 a ] \.{>} b}%
 \@x{\makebox[0pt][r]{\scriptsize 64\hspace{1em}}\@s{16.4} \.{\land}
 DidNotVoteAt ( a ,\, b )}%
\@pvspace{8.0pt}%
 \@x{\makebox[0pt][r]{\scriptsize 66\hspace{1em}} NoneOtherChoosableAt ( b ,\,
 v ) \.{\defeq}}%
 \@x{\makebox[0pt][r]{\scriptsize 67\hspace{1em}}\@s{16.4} \E\, Q \.{\in}
 Quorum \.{:}}%
 \@x{\makebox[0pt][r]{\scriptsize 68\hspace{1em}}\@s{27.72} \A\, a \.{\in} Q
 \.{:} VotedFor ( a ,\, b ,\, v ) \.{\lor} CannotVoteAt ( a ,\, b )}%
\@pvspace{8.0pt}%
 \@x{\makebox[0pt][r]{\scriptsize 70\hspace{1em}} SafeAt ( b ,\, v )
 \.{\defeq}}%
 \@x{\makebox[0pt][r]{\scriptsize 71\hspace{1em}}\@s{16.4} \A\, c \.{\in} 0
 \.{\dotdot} ( b \.{-} 1 ) \.{:} NoneOtherChoosableAt ( c ,\, v )}%
\@pvspace{8.0pt}%
\@x{\makebox[0pt][r]{\scriptsize 73\hspace{1em}} VotesSafe \.{\defeq}}%
 \@x{\makebox[0pt][r]{\scriptsize 74\hspace{1em}}\@s{16.4} \A\, a \.{\in}
 Acceptor ,\, b \.{\in} Ballot ,\, v \.{\in} Value \.{:}}%
 \@x{\makebox[0pt][r]{\scriptsize 75\hspace{1em}}\@s{27.72} VotedFor ( a ,\, b
 ,\, v ) \.{\implies} SafeAt ( b ,\, v )}%
\@pvspace{8.0pt}%
\@x{\makebox[0pt][r]{\scriptsize 77\hspace{1em}} OneVote \.{\defeq}}%
 \@x{\makebox[0pt][r]{\scriptsize 78\hspace{1em}}\@s{16.4} \A\, a \.{\in}
 Acceptor ,\, b \.{\in} Ballot ,\, v ,\, w \.{\in} Value \.{:}}%
 \@x{\makebox[0pt][r]{\scriptsize 79\hspace{1em}}\@s{27.72} VotedFor ( a ,\, b
 ,\, v ) \.{\land} VotedFor ( a ,\, b ,\, w ) \.{\implies} ( v \.{=} w )}%
\@pvspace{8.0pt}%
 \@x{\makebox[0pt][r]{\scriptsize 81\hspace{1em}} OneValuePerBallot
 \.{\defeq}}%
 \@x{\makebox[0pt][r]{\scriptsize 82\hspace{1em}}\@s{16.4} \A\, a1 ,\, a2
 \.{\in} Acceptor ,\, b \.{\in} Ballot ,\, v1 ,\, v2 \.{\in} Value \.{:}}%
 \@x{\makebox[0pt][r]{\scriptsize 83\hspace{1em}}\@s{27.72} VotedFor ( a1 ,\,
 b ,\, v1 ) \.{\land} VotedFor ( a2 ,\, b ,\, v2 ) \.{\implies} ( v1 \.{=} v2
 )}%
\@pvspace{8.0pt}%
 \@x{\makebox[0pt][r]{\scriptsize 85\hspace{1em}} Inv \.{\defeq} TypeOK
 \.{\land} VotesSafe \.{\land} OneValuePerBallot}%
\@x{\makebox[0pt][r]{\scriptsize 86\hspace{1em}}}\midbar\@xx{}%
 \@x{\makebox[0pt][r]{\scriptsize 87\hspace{1em}} {\THEOREM} AllSafeAtZero
 \.{\defeq} \A\, v \.{\in} Value \.{:} SafeAt ( 0 ,\, v )}%
\@x{\makebox[0pt][r]{\scriptsize 88\hspace{1em}}\@s{8.2} {\BY} {\DEF} SafeAt}%
\@pvspace{8.0pt}%
 \@x{\makebox[0pt][r]{\scriptsize 90\hspace{1em}} {\THEOREM} ChoosableThm
 \.{\defeq}}%
 \@x{\makebox[0pt][r]{\scriptsize 91\hspace{1em}}\@s{55.02} \A\, b \.{\in}
 Ballot ,\, v \.{\in} Value \.{:}}%
 \@x{\makebox[0pt][r]{\scriptsize 92\hspace{1em}}\@s{62.24} ChosenAt ( b ,\, v
 ) \.{\implies} NoneOtherChoosableAt ( b ,\, v )}%
 \@x{\makebox[0pt][r]{\scriptsize 93\hspace{1em}}\@s{8.2} {\BY} {\DEF}
 ChosenAt ,\, NoneOtherChoosableAt}%
\@pvspace{8.0pt}%
 \@x{\makebox[0pt][r]{\scriptsize 95\hspace{1em}} {\THEOREM} OneVoteThm
 \.{\defeq} OneValuePerBallot \.{\implies} OneVote}%
 \@x{\makebox[0pt][r]{\scriptsize 96\hspace{1em}}\@s{8.2} {\BY} {\DEF}
 OneValuePerBallot ,\, OneVote}%
\@x{\makebox[0pt][r]{\scriptsize 97\hspace{1em}}}\midbar\@xx{}%
 \@x{\makebox[0pt][r]{\scriptsize 98\hspace{1em}} {\THEOREM}
 VotesSafeImpliesConsistency \.{\defeq}}%
 \@x{\makebox[0pt][r]{\scriptsize 99\hspace{1em}}\@s{12.29} {\ASSUME}
 VotesSafe ,\, OneVote ,\, chosen \.{\neq} \{ \}}%
 \@x{\makebox[0pt][r]{\scriptsize 100\hspace{1em}}\@s{12.29} {\PROVE}\@s{4.1}
 \E\, v \.{\in} Value \.{:} chosen \.{=} \{ v \}}%
 \@x{\makebox[0pt][r]{\scriptsize 101\hspace{1em}}\@pfstepnum{1}{1.}\  {\PICK}
 v \.{\in} Value \.{:} v \.{\in} chosen}%
 \@x{\makebox[0pt][r]{\scriptsize 102\hspace{1em}}\@s{8.2} {\BY} {\DEF}
 chosen}%
 \@x{\makebox[0pt][r]{\scriptsize 103\hspace{1em}}\@pfstepnum{1}{2.}\ 
 {\SUFFICES} {\ASSUME} {\NEW} w \.{\in} chosen}%
 \@x{\makebox[0pt][r]{\scriptsize 104\hspace{1em}}\@s{68.17} {\PROVE}\@s{4.1}
 w \.{=} v}%
 \@x{\makebox[0pt][r]{\scriptsize 105\hspace{1em}}\@s{8.2}
 {\BY}\@pfstepnum{1}{1} ,\,\@pfstepnum{1}{2}\ }%
 \@x{\makebox[0pt][r]{\scriptsize 106\hspace{1em}}\@pfstepnum{1}{3.}\ 
 {\ASSUME} {\NEW} b1\@s{0.44} \.{\in} Ballot ,\, {\NEW} b2 \.{\in} Ballot ,\,
 b1 \.{<} b2 ,\,}%
 \@x{\makebox[0pt][r]{\scriptsize 107\hspace{1em}}\@s{62.13} {\NEW} v1 \.{\in}
 Value ,\, {\NEW} v2 \.{\in} Value ,\,}%
 \@x{\makebox[0pt][r]{\scriptsize 108\hspace{1em}}\@s{62.13} ChosenAt ( b1 ,\,
 v1 ) \.{\land} ChosenAt ( b2 ,\, v2 )}%
 \@x{\makebox[0pt][r]{\scriptsize 109\hspace{1em}}\@s{23.88} {\PROVE}\@s{4.1}
 v1 \.{=} v2}%
 \@x{\makebox[0pt][r]{\scriptsize 110\hspace{1em}}\@s{8.2}\@pfstepnum{2}{1.}\ 
 SafeAt ( b2 ,\, v2 )}%
 \@x{\makebox[0pt][r]{\scriptsize 111\hspace{1em}}\@s{16.4}
 {\BY}\@pfstepnum{1}{3} ,\, QuorumAssumption ,\, SMT {\DEF} ChosenAt ,\,
 VotesSafe}%
 \@x{\makebox[0pt][r]{\scriptsize 112\hspace{1em}}\@s{8.2}\@pfstepnum{2}{2.}\ 
 {\QED}}%
 \@x{\makebox[0pt][r]{\scriptsize 113\hspace{1em}}\@s{16.4}
 {\BY}\@pfstepnum{1}{3} ,\,\@pfstepnum{2}{1} ,\, QuorumAssumption ,\, Z3}%
 \@x{\makebox[0pt][r]{\scriptsize 114\hspace{1em}}\@s{16.4} {\DEFS}
 CannotVoteAt ,\, DidNotVoteAt ,\, OneVote ,\,}%
 \@x{\makebox[0pt][r]{\scriptsize 115\hspace{1em}}\@s{41.93} ChosenAt ,\,
 NoneOtherChoosableAt ,\, Ballot ,\, SafeAt}%
\@x{\makebox[0pt][r]{\scriptsize 116\hspace{1em}}\@pfstepnum{1}{4.}\  {\QED}}%
 \@x{\makebox[0pt][r]{\scriptsize 117\hspace{1em}}\@s{8.2} {\BY}
 QuorumAssumption ,\,\@pfstepnum{1}{1} ,\,\@pfstepnum{1}{2}
 ,\,\@pfstepnum{1}{3} ,\, Z3}%
 \@x{\makebox[0pt][r]{\scriptsize 118\hspace{1em}}\@s{8.2} {\DEFS} Ballot ,\,
 ChosenAt ,\, OneVote ,\, chosen}%
\@pvspace{8.0pt}%
 \@x{\makebox[0pt][r]{\scriptsize 120\hspace{1em}} {\THEOREM} ShowsSafety
 \.{\defeq}}%
 \@x{\makebox[0pt][r]{\scriptsize 121\hspace{1em}}\@s{55.02} TypeOK \.{\land}
 VotesSafe \.{\land} OneValuePerBallot \.{\implies}}%
 \@x{\makebox[0pt][r]{\scriptsize 122\hspace{1em}}\@s{67.32} \A\, Q \.{\in}
 Quorum ,\, b \.{\in} Ballot ,\, v \.{\in} Value \.{:}}%
 \@x{\makebox[0pt][r]{\scriptsize 123\hspace{1em}}\@s{75.52} ShowsSafeAt ( Q
 ,\, b ,\, v ) \.{\implies} SafeAt ( b ,\, v )}%
 \@x{\makebox[0pt][r]{\scriptsize 124\hspace{1em}}\@s{8.2} {\BY}
 QuorumAssumption ,\, Z3}%
 \@x{\makebox[0pt][r]{\scriptsize 125\hspace{1em}}\@s{8.2} {\DEFS} Ballot ,\,
 TypeOK ,\, VotesSafe ,\, OneValuePerBallot ,\, SafeAt ,\,}%
 \@x{\makebox[0pt][r]{\scriptsize 126\hspace{1em}}\@s{16.4} ShowsSafeAt ,\,
 CannotVoteAt ,\, NoneOtherChoosableAt ,\, DidNotVoteAt}%
\@pvspace{8.0pt}%
 \@x{\makebox[0pt][r]{\scriptsize 128\hspace{1em}} {\THEOREM} SafeAtStable
 \.{\defeq} Inv \.{\land} Next \.{\land} TypeOK \.{'} \.{\implies}}%
 \@x{\makebox[0pt][r]{\scriptsize 129\hspace{1em}}\@s{136.32} \A\, b \.{\in}
 Ballot ,\, v \.{\in} Value \.{:}}%
 \@x{\makebox[0pt][r]{\scriptsize 130\hspace{1em}}\@s{147.64} SafeAt ( b ,\, v
 ) \.{\implies} SafeAt ( b ,\, v ) \.{'}}%
\@x{\makebox[0pt][r]{\scriptsize 131\hspace{1em}}\@s{8.2} {\OMITTED}}%
\@x{\makebox[0pt][r]{\scriptsize 132\hspace{1em}}}\midbar\@xx{}%
 \@x{\makebox[0pt][r]{\scriptsize 133\hspace{1em}} {\THEOREM} Invariant
 \.{\defeq} Spec \.{\implies} {\Box} Inv}%
 \@x{\makebox[0pt][r]{\scriptsize 134\hspace{1em}}\@pfstepnum{1}{}\  {\USE}
 {\DEF} Inv}%
 \@x{\makebox[0pt][r]{\scriptsize 135\hspace{1em}}\@pfstepnum{1}{1.}\  Init
 \.{\implies} Inv}%
 \@x{\makebox[0pt][r]{\scriptsize 136\hspace{1em}}\@s{8.2} {\BY} {\DEF} Init
 ,\, TypeOK ,\, VotesSafe ,\, OneValuePerBallot ,\, VotedFor}%
 \@x{\makebox[0pt][r]{\scriptsize 137\hspace{1em}}\@pfstepnum{1}{2.}\  Inv
 \.{\land} [ Next ]_{ {\langle} votes ,\, maxBal {\rangle}} \.{\implies} Inv
 \.{'}}%
 \@x{\makebox[0pt][r]{\scriptsize 138\hspace{1em}}\@s{8.2}\@pfstepnum{2}{}\ 
 {\SUFFICES} {\ASSUME} Inv ,\, [ Next ]_{ {\langle} votes ,\, maxBal
 {\rangle}}}%
 \@x{\makebox[0pt][r]{\scriptsize 139\hspace{1em}}\@s{68.59} {\PROVE}\@s{4.1}
 Inv \.{'}}%
\@x{\makebox[0pt][r]{\scriptsize 140\hspace{1em}}\@s{16.4} {\OBVIOUS}}%
 \@x{\makebox[0pt][r]{\scriptsize 141\hspace{1em}}\@s{8.2}\@pfstepnum{2}{1.}
 {\CASE} Next}%
 \@x{\makebox[0pt][r]{\scriptsize 142\hspace{1em}}\@s{16.4}\@pfstepnum{3}{}\ 
 {\SUFFICES} {\ASSUME} {\NEW} a \.{\in} Acceptor ,\, {\NEW} b \.{\in} Ballot
 ,\,}%
 \@x{\makebox[0pt][r]{\scriptsize 143\hspace{1em}}\@s{115.04} \.{\lor}
 IncreaseMaxBal ( a ,\, b )}%
 \@x{\makebox[0pt][r]{\scriptsize 144\hspace{1em}}\@s{115.04} \.{\lor} \E\, v
 \.{\in} Value \.{:} VoteFor ( a ,\, b ,\, v )}%
 \@x{\makebox[0pt][r]{\scriptsize 145\hspace{1em}}\@s{76.79} {\PROVE}\@s{4.1}
 Inv \.{'}}%
 \@x{\makebox[0pt][r]{\scriptsize 146\hspace{1em}}\@s{24.59}
 {\BY}\@pfstepnum{2}{1}\  {\DEF} Next}%
 \@x{\makebox[0pt][r]{\scriptsize 147\hspace{1em}}\@s{16.4}\@pfstepnum{3}{1.}
 {\CASE} IncreaseMaxBal ( a ,\, b )}%
 \@x{\makebox[0pt][r]{\scriptsize
 148\hspace{1em}}\@s{24.59}\@pfstepnum{4}{1.}\  TypeOK \.{'}}%
 \@x{\makebox[0pt][r]{\scriptsize 149\hspace{1em}}\@s{32.8}
 {\BY}\@pfstepnum{3}{1}\  {\DEF} TypeOK ,\, IncreaseMaxBal}%
 \@x{\makebox[0pt][r]{\scriptsize
 150\hspace{1em}}\@s{24.59}\@pfstepnum{4}{2.}\  VotesSafe \.{'}}%
 \@x{\makebox[0pt][r]{\scriptsize 151\hspace{1em}}\@s{32.8}\@pfstepnum{5}{}\ 
 {\SUFFICES} {\ASSUME} {\NEW} a\_1 \.{\in} Acceptor \.{'} ,\, {\NEW} b\_1
 \.{\in} Ballot \.{'} ,\, {\NEW} v \.{\in} Value \.{'}}%
 \@x{\makebox[0pt][r]{\scriptsize 152\hspace{1em}}\@s{93.19} {\PROVE}\@s{4.1}
 VotedFor ( a\_1 ,\, b\_1 ,\, v ) \.{'} \.{\implies} SafeAt ( b\_1 ,\, v )
 \.{'}}%
 \@x{\makebox[0pt][r]{\scriptsize 153\hspace{1em}}\@s{41.0} {\BY} {\DEF}
 VotesSafe}%
 \@x{\makebox[0pt][r]{\scriptsize 154\hspace{1em}}\@s{32.8}\@pfstepnum{5}{1.}\
 \A\, aa \.{\in} Acceptor ,\, bb \.{\in} Ballot ,\, vv \.{\in} Value \.{:}}%
 \@x{\makebox[0pt][r]{\scriptsize 155\hspace{1em}}\@s{64.88} VotedFor ( aa ,\,
 bb ,\, vv ) \.{\equiv} VotedFor ( aa ,\, bb ,\, vv ) \.{'}}%
 \@x{\makebox[0pt][r]{\scriptsize 156\hspace{1em}}\@s{41.0}
 {\BY}\@pfstepnum{3}{1}\  {\DEF} IncreaseMaxBal ,\, VotedFor}%
 \@x{\makebox[0pt][r]{\scriptsize 157\hspace{1em}}\@s{32.8}\@pfstepnum{5}{2.}\
 \A\, aa \.{\in} Acceptor ,\, bb \.{\in} Ballot \.{:}}%
 \@x{\makebox[0pt][r]{\scriptsize 158\hspace{1em}}\@s{64.88} maxBal [ aa ]
 \.{>} bb \.{\implies} maxBal \.{'} [ aa ] \.{>} bb}%
 \@x{\makebox[0pt][r]{\scriptsize 159\hspace{1em}}\@s{41.0}
 {\BY}\@pfstepnum{3}{1}\  {\DEF} IncreaseMaxBal ,\, TypeOK ,\, Ballot}%
 \@x{\makebox[0pt][r]{\scriptsize 160\hspace{1em}}\@s{32.8}\@pfstepnum{5}{3.}\
 \A\, aa \.{\in} Acceptor ,\, bb \.{\in} Ballot \.{:}}%
 \@x{\makebox[0pt][r]{\scriptsize 161\hspace{1em}}\@s{64.88} DidNotVoteAt ( aa
 ,\, bb ) \.{\implies} DidNotVoteAt ( aa ,\, bb ) \.{'}}%
 \@x{\makebox[0pt][r]{\scriptsize 162\hspace{1em}}\@s{41.0}
 {\BY}\@pfstepnum{3}{1}\  {\DEF} IncreaseMaxBal ,\, DidNotVoteAt ,\,
 VotedFor}%
 \@x{\makebox[0pt][r]{\scriptsize 163\hspace{1em}}\@s{32.8}\@pfstepnum{5}{4.}\
 \A\, aa \.{\in} Acceptor ,\, bb \.{\in} Ballot \.{:}}%
 \@x{\makebox[0pt][r]{\scriptsize 164\hspace{1em}}\@s{64.88} CannotVoteAt ( aa
 ,\, bb ) \.{\implies} CannotVoteAt ( aa ,\, bb ) \.{'}}%
 \@x{\makebox[0pt][r]{\scriptsize 165\hspace{1em}}\@s{41.0}
 {\BY}\@pfstepnum{3}{1} ,\,\@pfstepnum{5}{2} ,\,\@pfstepnum{5}{3}\  {\DEF}
 IncreaseMaxBal ,\, CannotVoteAt}%
 \@x{\makebox[0pt][r]{\scriptsize 166\hspace{1em}}\@s{32.8}\@pfstepnum{5}{5.}\
 \A\, bb \.{\in} Ballot ,\, vv \.{\in} Value \.{:}}%
 \@x{\makebox[0pt][r]{\scriptsize 167\hspace{1em}}\@s{64.88}
 NoneOtherChoosableAt ( bb ,\, vv ) \.{\implies} NoneOtherChoosableAt ( bb
 ,\, vv ) \.{'}}%
 \@x{\makebox[0pt][r]{\scriptsize 168\hspace{1em}}\@s{41.0}
 {\BY}\@pfstepnum{5}{1} ,\,\@pfstepnum{5}{4} ,\, QuorumAssumption {\DEFS}
 NoneOtherChoosableAt}%
 \@x{\makebox[0pt][r]{\scriptsize 169\hspace{1em}}\@s{32.8}\@pfstepnum{5}{6.}\
 {\QED}}%
 \@x{\makebox[0pt][r]{\scriptsize 170\hspace{1em}}\@s{41.0}
 {\BY}\@pfstepnum{5}{1} ,\,\@pfstepnum{5}{5}\  {\DEF} TypeOK ,\, Ballot ,\,
 VotesSafe ,\, SafeAt}%
 \@x{\makebox[0pt][r]{\scriptsize
 171\hspace{1em}}\@s{24.59}\@pfstepnum{4}{3.}\  OneValuePerBallot \.{'}}%
 \@x{\makebox[0pt][r]{\scriptsize 172\hspace{1em}}\@s{32.8}
 {\BY}\@pfstepnum{3}{1}\  {\DEF} IncreaseMaxBal ,\, OneValuePerBallot ,\,
 VotedFor}%
 \@x{\makebox[0pt][r]{\scriptsize
 173\hspace{1em}}\@s{24.59}\@pfstepnum{4}{4.}\  {\QED}}%
 \@x{\makebox[0pt][r]{\scriptsize 174\hspace{1em}}\@s{32.8}
 {\BY}\@pfstepnum{4}{1} ,\,\@pfstepnum{4}{2} ,\,\@pfstepnum{4}{3}\  {\DEF}
 Inv}%
 \@x{\makebox[0pt][r]{\scriptsize 175\hspace{1em}}\@s{16.4}\@pfstepnum{3}{2.}\
 {\ASSUME} {\NEW} v \.{\in} Value ,\,}%
 \@x{\makebox[0pt][r]{\scriptsize 176\hspace{1em}}\@s{78.53} VoteFor ( a ,\, b
 ,\, v )}%
 \@x{\makebox[0pt][r]{\scriptsize 177\hspace{1em}}\@s{40.28} {\PROVE}\@s{4.1}
 Inv \.{'}}%
 \@x{\makebox[0pt][r]{\scriptsize 178\hspace{1em}}\@s{24.59}\@pfstepnum{4}{}\ 
 {\SUFFICES} {\ASSUME} {\NEW} Q \.{\in} Quorum ,\,}%
 \@x{\makebox[0pt][r]{\scriptsize 179\hspace{1em}}\@s{123.24} ShowsSafeAt ( Q
 ,\, b ,\, v )}%
 \@x{\makebox[0pt][r]{\scriptsize 180\hspace{1em}}\@s{84.99} {\PROVE}\@s{4.1}
 Inv \.{'}}%
 \@x{\makebox[0pt][r]{\scriptsize 181\hspace{1em}}\@s{32.8}
 {\BY}\@pfstepnum{3}{2}\  {\DEF} VoteFor}%
 \@x{\makebox[0pt][r]{\scriptsize
 182\hspace{1em}}\@s{24.59}\@pfstepnum{4}{1.}\  TypeOK \.{'}}%
 \@x{\makebox[0pt][r]{\scriptsize 183\hspace{1em}}\@s{32.8}
 {\BY}\@pfstepnum{3}{2}\  {\DEF} TypeOK ,\, VoteFor}%
 \@x{\makebox[0pt][r]{\scriptsize
 184\hspace{1em}}\@s{24.59}\@pfstepnum{4}{2.}\  VotesSafe \.{'}}%
\@y{\@s{0}%
 Using \ensuremath{OneValuePerBallot} in \ensuremath{SafeAtStable
}}%
\@xx{}%
 \@x{\makebox[0pt][r]{\scriptsize 185\hspace{1em}}\@s{32.8}\@pfstepnum{5}{}\ 
 {\SUFFICES} {\ASSUME} {\NEW} aa \.{\in} Acceptor \.{'} ,\, {\NEW} bb \.{\in}
 Ballot \.{'} ,\, {\NEW} vv \.{\in} Value \.{'} ,\,}%
 \@x{\makebox[0pt][r]{\scriptsize 186\hspace{1em}}\@s{131.44} VotedFor ( aa
 ,\, bb ,\, vv ) \.{'}}%
 \@x{\makebox[0pt][r]{\scriptsize 187\hspace{1em}}\@s{93.19} {\PROVE}\@s{4.1}
 SafeAt ( bb ,\, vv ) \.{'}}%
 \@x{\makebox[0pt][r]{\scriptsize 188\hspace{1em}}\@s{41.0} {\BY} {\DEF}
 VotesSafe}%
 \@x{\makebox[0pt][r]{\scriptsize 189\hspace{1em}}\@s{32.8}\@pfstepnum{5}{1.}
 {\CASE} VotedFor ( aa ,\, bb ,\, vv )}%
 \@x{\makebox[0pt][r]{\scriptsize 190\hspace{1em}}\@s{41.0}\@pfstepnum{6}{1.}\
 SafeAt ( bb ,\, vv )}%
 \@x{\makebox[0pt][r]{\scriptsize 191\hspace{1em}}\@s{49.2}
 {\BY}\@pfstepnum{5}{1}\  {\DEF} VotesSafe}%
 \@x{\makebox[0pt][r]{\scriptsize 192\hspace{1em}}\@s{41.0}\@pfstepnum{6}{}\ 
 {\QED}}%
 \@x{\makebox[0pt][r]{\scriptsize 193\hspace{1em}}\@s{49.2}
 {\BY}\@pfstepnum{4}{1} ,\,\@pfstepnum{6}{1} ,\, SafeAtStable {\DEF} Next}%
 \@x{\makebox[0pt][r]{\scriptsize 194\hspace{1em}}\@s{32.8}\@pfstepnum{5}{2.}
 {\CASE} {\lnot} VotedFor ( aa ,\, bb ,\, vv )}%
 \@x{\makebox[0pt][r]{\scriptsize 195\hspace{1em}}\@s{41.0}\@pfstepnum{6}{1.}\
 aa \.{=} a \.{\land} bb \.{=} b \.{\land} vv \.{=} v \.{\land} VotedFor ( a
 ,\, b ,\, v ) \.{'}}%
 \@x{\makebox[0pt][r]{\scriptsize 196\hspace{1em}}\@s{49.2}
 {\BY}\@pfstepnum{3}{2} ,\,\@pfstepnum{4}{1} ,\,\@pfstepnum{5}{2}\  {\DEF}
 VoteFor ,\, VotedFor ,\, TypeOK}%
 \@x{\makebox[0pt][r]{\scriptsize 197\hspace{1em}}\@s{41.0}\@pfstepnum{6}{}\ 
 {\QED}}%
 \@x{\makebox[0pt][r]{\scriptsize 198\hspace{1em}}\@s{49.2}
 {\BY}\@pfstepnum{4}{1} ,\,\@pfstepnum{6}{1} ,\, ShowsSafety ,\, SafeAtStable
 {\DEF} VoteFor ,\, Next}%
 \@x{\makebox[0pt][r]{\scriptsize 199\hspace{1em}}\@s{32.8}\@pfstepnum{5}{}\ 
 {\QED}}%
 \@x{\makebox[0pt][r]{\scriptsize 200\hspace{1em}}\@s{41.0}
 {\BY}\@pfstepnum{5}{1} ,\,\@pfstepnum{5}{2}\ }%
 \@x{\makebox[0pt][r]{\scriptsize
 201\hspace{1em}}\@s{24.59}\@pfstepnum{4}{3.}\  OneValuePerBallot \.{'}}%
 \@x{\makebox[0pt][r]{\scriptsize 202\hspace{1em}}\@s{32.8}
 {\BY}\@pfstepnum{3}{2}\  {\DEF} VoteFor ,\, OneValuePerBallot ,\, VotedFor
 ,\, TypeOK}%
 \@x{\makebox[0pt][r]{\scriptsize
 203\hspace{1em}}\@s{24.59}\@pfstepnum{4}{4.}\  {\QED}}%
 \@x{\makebox[0pt][r]{\scriptsize 204\hspace{1em}}\@s{32.8}
 {\BY}\@pfstepnum{3}{2} ,\,\@pfstepnum{4}{1} ,\,\@pfstepnum{4}{2}
 ,\,\@pfstepnum{4}{3}\  {\DEF} Inv}%
 \@x{\makebox[0pt][r]{\scriptsize 205\hspace{1em}}\@s{16.4}\@pfstepnum{3}{3.}\
 {\QED}}%
 \@x{\makebox[0pt][r]{\scriptsize 206\hspace{1em}}\@s{24.59}
 {\BY}\@pfstepnum{2}{1} ,\,\@pfstepnum{3}{1} ,\,\@pfstepnum{3}{2}\ }%
 \@x{\makebox[0pt][r]{\scriptsize 207\hspace{1em}}\@s{8.2}\@pfstepnum{2}{2.}
 {\CASE} {\UNCHANGED} {\langle} votes ,\, maxBal {\rangle}}%
 \@x{\makebox[0pt][r]{\scriptsize 208\hspace{1em}}\@s{16.4}
 {\BY}\@pfstepnum{2}{2}\ }%
 \@x{\makebox[0pt][r]{\scriptsize 209\hspace{1em}}\@s{16.4} {\DEFS} TypeOK ,\,
 Next ,\, VotesSafe ,\, OneValuePerBallot ,\,}%
 \@x{\makebox[0pt][r]{\scriptsize 210\hspace{1em}}\@s{41.93} VotedFor ,\,
 SafeAt ,\, NoneOtherChoosableAt ,\, CannotVoteAt ,\, DidNotVoteAt ,\,}%
 \@x{\makebox[0pt][r]{\scriptsize 211\hspace{1em}}\@s{41.93} IncreaseMaxBal
 ,\, VoteFor}%
 \@x{\makebox[0pt][r]{\scriptsize 212\hspace{1em}}\@s{8.2}\@pfstepnum{2}{3.}\ 
 {\QED}}%
 \@x{\makebox[0pt][r]{\scriptsize 213\hspace{1em}}\@s{16.4}
 {\BY}\@pfstepnum{2}{1} ,\,\@pfstepnum{2}{2}\ }%
\@x{\makebox[0pt][r]{\scriptsize 214\hspace{1em}}\@pfstepnum{1}{3.}\  {\QED}}%
 \@x{\makebox[0pt][r]{\scriptsize 215\hspace{1em}}\@s{8.2}
 {\BY}\@pfstepnum{1}{1} ,\,\@pfstepnum{1}{2} ,\, PTL {\DEF} Spec}%
\@x{\makebox[0pt][r]{\scriptsize 216\hspace{1em}}}\midbar\@xx{}%
 \@x{\makebox[0pt][r]{\scriptsize 217\hspace{1em}} {\THEOREM} Consistent
 \.{\defeq} Spec \.{\implies} {\Box} Consistency}%
 \@x{\makebox[0pt][r]{\scriptsize 218\hspace{1em}}\@pfstepnum{1}{}\  {\USE}
 {\DEF} Ballot}%
 \@x{\makebox[0pt][r]{\scriptsize 219\hspace{1em}}\@pfstepnum{1}{1.}\  Inv
 \.{\implies} Consistency}%
 \@x{\makebox[0pt][r]{\scriptsize 220\hspace{1em}}\@s{8.2}\@pfstepnum{2}{}\ 
 {\SUFFICES} {\ASSUME} Inv}%
 \@x{\makebox[0pt][r]{\scriptsize 221\hspace{1em}}\@s{68.59} {\PROVE}\@s{4.1}
 Consistency}%
\@x{\makebox[0pt][r]{\scriptsize 222\hspace{1em}}\@s{16.4} {\OBVIOUS}}%
 \@x{\makebox[0pt][r]{\scriptsize 223\hspace{1em}}\@s{8.2}\@pfstepnum{2}{}\ 
 {\QED}}%
 \@x{\makebox[0pt][r]{\scriptsize 224\hspace{1em}}\@s{16.4} {\BY}
 VotesSafeImpliesConsistency ,\, OneVoteThm {\DEF} Inv ,\, Consistency}%
\@x{\makebox[0pt][r]{\scriptsize 225\hspace{1em}}\@pfstepnum{1}{2.}\  {\QED}}%
 \@x{\makebox[0pt][r]{\scriptsize 226\hspace{1em}}\@s{8.2} {\BY} Invariant
 ,\,\@pfstepnum{1}{1} ,\, PTL}%
\@x{\makebox[0pt][r]{\scriptsize 227\hspace{1em}}}\midbar\@xx{}%
 \@x{\makebox[0pt][r]{\scriptsize 228\hspace{1em}} C \.{\defeq} {\INSTANCE}
 Consensus}%
\@y{\@s{0}%
 \ensuremath{{\WITH}} \ensuremath{chosen \.{\leftarrow} chosen
}}%
\@xx{}%
\@pvspace{8.0pt}%
 \@x{\makebox[0pt][r]{\scriptsize 230\hspace{1em}} {\THEOREM} Refinement
 \.{\defeq} Spec \.{\implies} C {\bang} Spec}%
 \@x{\makebox[0pt][r]{\scriptsize 231\hspace{1em}}\@pfstepnum{1}{1.}\  Init
 \.{\implies} C {\bang} Init}%
 \@x{\makebox[0pt][r]{\scriptsize 232\hspace{1em}}\@s{8.2} {\BY}
 QuorumAssumption ,\, SetExtensionality ,\, IsaM (\@w{force} )}%
 \@x{\makebox[0pt][r]{\scriptsize 233\hspace{1em}}\@s{8.2} {\DEF} Init ,\, C
 {\bang} Init ,\, chosen ,\, ChosenAt ,\, VotedFor}%
 \@x{\makebox[0pt][r]{\scriptsize 234\hspace{1em}}\@pfstepnum{1}{2.}\  TypeOK
 \.{'} \.{\land} Consistency \.{'} \.{\land} [ Next ]_{ {\langle} votes ,\,
 maxBal {\rangle}} \.{\implies} [ C {\bang} Next ]_{ chosen}}%
 \@x{\makebox[0pt][r]{\scriptsize 235\hspace{1em}}\@s{8.2}\@pfstepnum{2}{1.}\ 
 {\UNCHANGED} {\langle} votes ,\, maxBal {\rangle} \.{\implies} {\UNCHANGED}
 chosen}%
 \@x{\makebox[0pt][r]{\scriptsize 236\hspace{1em}}\@s{16.4} {\BY} {\DEF}
 chosen ,\, ChosenAt ,\, VotedFor}%
 \@x{\makebox[0pt][r]{\scriptsize 237\hspace{1em}}\@s{8.2}\@pfstepnum{2}{2.}\ 
 TypeOK \.{'} \.{\land} Consistency \.{'} \.{\land} Next \.{\implies} C
 {\bang} Next \.{\lor} {\UNCHANGED} chosen}%
 \@x{\makebox[0pt][r]{\scriptsize 238\hspace{1em}}\@s{16.4}\@pfstepnum{3}{1.}\
 {\SUFFICES} {\ASSUME} TypeOK \.{'} ,\, Consistency \.{'} ,\, Next}%
 \@x{\makebox[0pt][r]{\scriptsize 239\hspace{1em}}\@s{84.57} {\PROVE}\@s{4.1}
 C {\bang} Next \.{\lor} {\UNCHANGED} chosen}%
\@x{\makebox[0pt][r]{\scriptsize 240\hspace{1em}}\@s{24.59} {\OBVIOUS}}%
 \@x{\makebox[0pt][r]{\scriptsize 241\hspace{1em}}\@s{16.4}\@pfstepnum{3}{2.}\
 chosen \.{\subseteq} chosen \.{'}}%
 \@x{\makebox[0pt][r]{\scriptsize 242\hspace{1em}}\@s{24.59}
 {\BY}\@pfstepnum{3}{1} ,\, QuorumAssumption ,\, Z3}%
 \@x{\makebox[0pt][r]{\scriptsize 243\hspace{1em}}\@s{24.59} {\DEFS} Next ,\,
 IncreaseMaxBal ,\, VoteFor ,\, Inv ,\, TypeOK ,\, chosen ,\, ChosenAt ,\,
 VotedFor ,\, Ballot}%
 \@x{\makebox[0pt][r]{\scriptsize 244\hspace{1em}}\@s{16.4}\@pfstepnum{3}{3.}\
 chosen \.{'} \.{=} \{ \} \.{\lor} \E\, v \.{\in} Value \.{:} chosen \.{'}
 \.{=} \{ v \}}%
 \@x{\makebox[0pt][r]{\scriptsize 245\hspace{1em}}\@s{24.59}
 {\BY}\@pfstepnum{3}{1}\  {\DEF} Consistency}%
 \@x{\makebox[0pt][r]{\scriptsize 246\hspace{1em}}\@s{16.4}\@pfstepnum{3}{4.}\
 {\QED}}%
 \@x{\makebox[0pt][r]{\scriptsize 247\hspace{1em}}\@s{24.59}
 {\BY}\@pfstepnum{3}{1} ,\,\@pfstepnum{3}{2} ,\,\@pfstepnum{3}{3}\  {\DEF} C
 {\bang} Next}%
 \@x{\makebox[0pt][r]{\scriptsize 248\hspace{1em}}\@s{8.2}\@pfstepnum{2}{3.}\ 
 {\QED}}%
 \@x{\makebox[0pt][r]{\scriptsize 249\hspace{1em}}\@s{16.4}
 {\BY}\@pfstepnum{2}{1} ,\,\@pfstepnum{2}{2}\ }%
\@x{\makebox[0pt][r]{\scriptsize 250\hspace{1em}}\@pfstepnum{1}{3.}\  {\QED}}%
 \@x{\makebox[0pt][r]{\scriptsize 251\hspace{1em}}\@s{8.2}
 {\BY}\@pfstepnum{1}{1} ,\,\@pfstepnum{1}{2} ,\, Invariant ,\, Consistent ,\,
 PTL {\DEF} Spec ,\, Inv ,\, C {\bang} Spec}%
\@x{\makebox[0pt][r]{\scriptsize 252\hspace{1em}}}\bottombar\@xx{}%
\end{document}
